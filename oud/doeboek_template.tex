% Template voor een doeboekje van de
% Stichting Vierkant voor Wiskunde. Copyright 2000-2003:
% M.P. Alberts, J.C.C. Langeveld, M. Hendriks, W.J. Palenstijn.
% Meest recente update: 07-03-2003

% De 'beperklegenda' optie zorgt ervoor dat alleen de symbolen die
% echt gebruikt worden in de legenda komen.
% De 'geenantwoorden' optie laat de antwoorden weg (voor uiteindelijke versie)
% De 'geendatum' optie laat de datum weg (voor uiteindelijke versie)
% \documentclass[beperklegenda,geenantwoorden,geendatum]{doeboek}
\documentclass[beperklegenda]{doeboek}

% ------- Vul hier de gegevens van het doeboekje in
\auteur{Maurice Alberts\\
Maxim Hendriks\\
Joost Langeveld\\
Willem Jan Palenstijn}
\setcopyright{2002 M. Alberts, M. Hendriks, J. Langeveld, W.J. Palenstijn}
\illustrator{Niemand}
\titel{Doeboek Template}
\nummer{37}
\illustratie{vierkantlogo}
% -------

\begin{document}
\voorpagina
\wistjedat

% ------- Zet hier een aantal leuke wetenswaardigheden over het onderwerp
\ldots Hier de leuke wetenswaardigheden
% -------

Lees dit doe-boek om nog veel meer te weten te komen!
\CWIinfo

\titelpagina

\voorwoord
% ------- Schrijf hier een voorwoord
Hier het voorwoord van het doeboek
% -------

\legenda
\inhoudsopgave

% ------- Hier begint de eigenlijke tekst van het doeboek
\chapter{Het eerste hoofdstuk}
Hier de hoofdtekst van het doeboek.

\begin{opgave}[\ster\vinger\schaar\gr,5]
Vul deze tekst aan tot een volledig doeboek
\begin{hint}
Dit is meer werk dan je zou denken!
\end{hint}
\end{opgave}
% ------- Hier eindigt de eigenlijke tekst


\hintsantwoorden

\vierkantuitleg

In de reeks Vierkant doeboeken verschenen reeds:

\begin{tabular}{rl}
1.  & Spelen op een slimme manier {\aut Zs. Ruttkay}\\
2.  & Tegels leggen, en dergelijke {\aut F. G\"obel}\\
3.  & Veelvlakken {\aut A.W. Grootendorst, M.J.H. Pijls, Zs. Ruttkay}\\
4.  & Kun je deze oplossen? {\aut L. Barendregt, W. Oudshoorn, Zs. Ruttkay}\\
5.  & Van plattegrond tot scharrelkip {\aut F. G\"obel}\\
6.  & W.A. Mozart, Muzikaal Dobbelspel I, II {\aut M. Pijls, Zs. Ruttkay}\\
7.  & Fibonacci-getallen en de gulden snede {\aut Zs. Ruttkay, E.C. Buissant des Amorie}\\
8.  & Magische vierkanten {\aut A.A.J. Lefeber}\\
9.  & De veelzijdigheid van vierkanten {\aut R. Roelofs, F. G\"obel}\\
10. & Durf je deze aan te pakken? {\aut C. Wildhagen}\\
11. & Knopen {\aut F. Akveld, M. Akveld}\\
12. & Wiskunde kalender 1998 {\aut R. Roelofs, F. G\"obel}\\
13. & Met passer en latje {\aut Zs. Ruttkay}\\
14. & Getallen {\aut A. Heinis, H. Brandsma}\\
15. & Wiskunde kalender 1999 {\aut F. G\"obel, R. Roelofs}\\
16. & Puzzels voor junioren {\aut C. Wildhagen}\\
17. & Zelf Escher-achtige vlakvullingen ontwerpen {\aut A. Kolkman, M. Pijls}\\
18. & Het getal $\pi$ {\aut E.C. Buissant des Amorie}\\
19. & Lijnen in perspectief {\aut F. van der Blij, A. Carsouw}\\
20. & Wiskunde kalender 2000 {\aut R. Roelofs, F. G�bel}\\
21. & Kijk op kegelsneden {\aut L. de Clerck, R. van der Waall}\\
22. & Cryptologie {\aut M.P. Alberts, J.C.C. Langeveld}\\
% ------- Voer hier eventueel nieuw verschenen doeboeken toe (inclusief deze!)
% -------
\end{tabular}

\bestelinfo

\end{document}
