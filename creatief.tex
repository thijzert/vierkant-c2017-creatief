\documentclass{weekprobleem}
\usepackage[utf8]{inputenc}
\begin{document}
\weekprobleem{2017}{C}{Creatieve weekproblemen: scheepvaart}

Dit jaar hebben de creatieve weekproblemen het thema scheepvaart, [en daar is vast een wat meer literaire inleiding voor te verzinnen].

\section*{Vouw een boot}

Vrijwel iedereen weet hoe je van een A5'je een bootje kunt vouwen, maar bootje dat je krijgt is op groter formaat een beetje saai.

De opdracht: neem een vel A3-papier en vouw een bootje, en probeer er zo veel mogelijk details in te origamiën.
Denk hierbij bijvoorbeeld de hut van de kapitein, een zeemeerminnenboegbeeld, of een radar-antenne.

Let op: zoals gebruikelijk bij origami zijn de bouwwerken gemaakt van één stuk papier, en zonder schaar of plakband\footnote{Goed verstoppen dus!}.


\section*{Maak een sterrenbeeld}

Tegenwoordig heeft iedereen een GPS-antenne in zijn of haar broekzak, maar honderd jaar geleden ging dat héél anders: schepen die de oceanen afstruinden moesten navigeren op de sterren.

[bruggetje]

Neem je favoriete irrationele getal ($\pi$, $e$, $\sqrt{2}$, etc.), en verzin een manier om van de decimalen een sterrenbeeld van te maken.
Je kunt bijvoorbeeld de decimalen in een coördinatensysteem tekenen en er lijntjes tussen trekken, of [iets met graden naar rechts plotten].

Wat beeldt je lijnentekening uit?\\
Wat is het mythologische verhaal achter jouw sterrenbeeld?


\section*{Gestrand!}

Helaas, het sterrenbeeld van hierboven bleek toch niet betrouwbaar genoeg om de weg te vinden, en de boot van nog eerder was niet stevig genoeg om de botsing met dit onbewoonde eiland te weerstaan. Je bent gestrand!

Als je vier getallen mocht meenemen naar een onbewoond eiland om je gezelschap te houden, welke zou je dan kiezen? Waarom?

Schrijf een verhaal over de avonturen die jullie met zijn vijven beleven.

\end{document} 
